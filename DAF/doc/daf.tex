\documentclass[preprint]{revtex4}

%%%%%%%%%%%%%%%%%%%%%%%%%%%%%%%%%%%%%%%%%%%%%%%%%%%%%%%%%%%%%%%%%%%%%%%%%%%%%%%%%%%%%%%%%%
\usepackage{amsmath}
\newcommand{\rtpi}{\ensuremath{\sqrt{\pi}}}
\newcommand{\intl}{\int\displaylimits}
\newcommand{\intunit}{\ensuremath{\int_{-1}^{1}}}
\newcommand{\sumunit}{\ensuremath{\sum_{i=1}^{n}}}
\newcommand{\dx}{{\rm d}x}
\newcommand{\dy}{{\rm d}y}
\newcommand{\wbar}{\ensuremath{\bar{w}}}

%%%%%%%%%%%%%%%%%%%%%%%%%%%%%%%%%%%%%%%%%%%%%%%%%%%%%%%%%%%%%%%%%%%%%%%%%%%%%%%%%%%%%%%%%%
\begin{document}

\title{%
    Distributed Approximating Functionals\\
    %\vspace{0.5cm}
    {\small Theory and Computational Methodology}\\
}
\author{Daniel A. Brue}
%\email{danielbrue (at) Gmail}
\date{\today}

\begin{abstract}
The abstract goes here after the paper is done.
\end{abstract}

\maketitle

%%%%%%%%%%%%%%%%%%%%%%%%%%%%%%%%%%%%%%%%%%%%%%%%%%%%%%%%%%%%%%%%%%%%%%%%%%%%%%%%%%%%%%%%%%
\section{Introduction}
The distributed approximating functionals (DAF) method was developed 
and implemented for solving classes of differential equations very accurately
with a high fidelity to the accuracy of derivatives. 
The DAF method was applied first to time-dependent propgation problems in theoretical
quantum physics applications\cite{Kouri-DAF1,Kouri-DAF2,Kouri-DAF3,Kouri-DAF4,
Kouri-DAF5,Kouri-DAF6,Kouri-DAF7}.

The DAF method has been used in various situations that involve time evolution, 
from quantum particle trajectories\cite{QTM}, integration of Feynman path integrals\cite{DAF-Szalay1},
and fluid dynamics\cite{NVS,Burgers}, and also as a method of solving bound eigenvalue-eigenvector
problems\cite{DAF-Szalay2,SGWD} and data interpolation\cite{DAF-Szalay3}.

The DAF method is more computationally intensive in practice than other simpler methods, 
however, much can be precomputed and reused, so the most time consuming aspects of the
method need be computed only once 

\begin{enumerate}
\item{\bf Differential Equations}: The DAF method constructs matrix operators that
	can be applied to multidimentional linear differential equations. 
\item{\bf Interpolation}: The DAF method can be used to approximate functions and
	their derivatives with high accuracy. 
\end{enumerate}

%-----------------------------------------------------------------------------------------

%%%%%%%%%%%%%%%%%%%%%%%%%%%%%%%%%%%%%%%%%%%%%%%%%%%%%%%%%%%%%%%%%%%%%%%%%%%%%%%%%%%%%%%%%%
\section{Mathematical Backgound}
In this section, we develop the mathematical foundation for the DAF method, establish 
the needed relations to employ the DAF in practice, and define its limitations. 

%-----------------------------------------------------------------------------------------
\subsection{Hermite Expansion of Delta Functions}
The DAF method uses an approximation of the Dirac delta function generated by 
an expansion in Hermite polynomials. With the appropriate Gaussian weight function, 
the Hermite polynomials form a complete and orthogonal basis on the domain of 
$(-\infty,\infty)$. 
This basis set can be used to expand any function on this domain. The advantages
of this method include high degrees of accuracy in representing other functions, 
and provides an analytic, functional approximation of undefined functions such as 
what might be represented by discrete data points. 

The expansion follows the general formula\cite{HermiteWiki} of 
\begin{equation}
\label{eqn_deltaHexp0}
\delta(x) = \sum_{m=0}^\infty h_m H_m\left(\frac{x}{\sigma}\right) e^{-\left(\frac{x}{\sigma}\right)}
\end{equation}
Where $H_m$ are the Hermite polynomials, $\sigma$ is a scaling function, and $h_m$ are the
expansion coefficients. A few notes on these terms: 
\begin{itemize}
\item{$H_m(x)$}: There are two common definitions for the Hermite polynomials, which are referred
to commonly as the {\it probability} and the {\it physics} definitions, which here we follow
Wikipedia's notation and use $H_m$ for the physics definition and ${\it He}_m$ for the probabilistic. 
They are related by ${\it He}_m = 2^m H_m$. In this document, we use the physics definition. 
\item{$\sigma$}: The scaling function $\sigma$ can be used to scale the Hermite functions 
``horizontally''. In the practical case in which the infinite sum in equation \ref{eqn_deltaHexp0}
is limited to a finite number of terms, the $\sigma$ factor directly effects how many terms
are needed to represent a function. 
A poorly chosen value of $\sigma$ can make converging the summation very difficult. 

\item{$h_m$}: The expansion coefficients represent how much each Hermite term contributes to the
expansion. These can be derived analytically for the $\delta(x)$ expansion.
This derivation is shown below. 
\end{itemize}

To determine the expansion coefficients $h_m$, we begin with equation \ref{eqn_deltaHexp0}
and multiply each side by $H_n(x/\sigma)$ and integrate over $(-\infty,\infty)$, 
\begin{eqnarray}
\label{eqn_deltaHexp1}
\intl_{-\infty}^{\infty}H_n\left(\frac{x}{\sigma}\right)\delta(x)dx & = & \intl_{-\infty}^{\infty}H_n\left(\frac{x}{\sigma}\right)\sum_{m=0}^\infty h_m H_m\left(\frac{x}{\sigma}\right) e^{-\left(\frac{x}{\sigma}\right)} dx \\*
H_n(0) & = & 
\sum_{m=0}^\infty h_m \intl_{-\infty}^{\infty}H_n\left(\frac{x}{\sigma}\right)
H_m\left(\frac{x}{\sigma}\right) e^{-\left(\frac{x}{\sigma}\right)} dx 
\label{eqn_deltaHexp2}
\end{eqnarray}
Next we use the orthogonality property\cite{HermiteWiki,HermitePolyMW} of 
\begin{equation}
\label{eqn_Hortho}
\intl_{-\infty}^{\infty}H_m(x)H_n(x)e^{-x^2}dx = \rtpi\, 2^m m!\, \delta_{mn}
\end{equation}
where $\delta_{mn}$ is the Kronecker Delta function and is equal to $1$ if $m=n$ and $0$ otherwise. 
With a scaling factor, $\sigma$, this becomes
\begin{equation}
\label{eqn_Hortho1}
%\intl_{-\infty}^{\infty}H_m\left(\frac{x}{\sigma}x\right)H_n\left(\frac{x}{\sigma}\right)e^{-\left(\frac{x}{\sigma}x\right)^2dx = \rtpi 2^m m! \delta_{mn}
\intl_{-\infty}^{\infty} H_m\left(\frac{x}{\sigma}\right) H_n\left(\frac{x}{\sigma}\right) 
e^{-\left(\frac{x}{\sigma}\right)^2}dx = 
\sigma \rtpi\, 2^m m!\, \delta_{mn}
\end{equation}

Using this in equation \ref{eqn_deltaHexp2}, we have

\begin{equation*}
H_n(0)  =  \sum_{m=0}^\infty h_m \sigma \rtpi\, 2^m m!\, \delta_{mn} \nonumber
\end{equation*}
which gives
\begin{equation}
h_n = \frac{H_n(0)}{\sigma \rtpi \, 2^n \, n!}
\label{eqn_deltaHexp_hn}
\end{equation}
One thing that can be observed by this definition is that for all odd values of $n$, the
coefficient $h_n$ is zero because $H_n(0)=0$ for all odd-order Hermite polynomials. 
This makes sense when we consider that the Dirac delta function is symmetric in $x$, and
therefore only even Hermite functions, i.e. those that are also symmetric in $x$, will
contribute to the expansion series. 

Note that we can use the Hermite polynomial recurrsion relation, given as
\begin{equation}
\label{eqn_HermiteRecur}
H_{n+1}(x) = 2xH_n(x)-2nH_{n-1}(x)
\end{equation}
to quickly find the coefficients. We can see that the first several $h_n$  are
\begin{eqnarray}
h_0 & =&  1 \nonumber \\
h_1 &=& 0 \nonumber \\
h_2 &=& 2*0*H_1(0)-2*1*H_0(0) = -2 \nonumber \\
h_3 &=& 0 \nonumber \\
h_4 &=& 2*0*H_3(0) - 2*3*-2= 12 \nonumber \\
&...& \nonumber
\end{eqnarray}

From this we define a function, $\Delta_{\sigma,M}(x)$ that depends parametrically on
$\sigma$ and $M$, and is exact as $M\to\infty$, 
\begin{equation}
\label{eqn_DeltaExpansion}
\Delta_{\sigma,M}(x) = \sum_m^M h_m H_m \left(\frac{x}{\sigma}\right)e^{- \left(\frac{x}{\sigma}\right)^2}
\end{equation}

%-----------------------------------------------------------------------------------------

%%%%%%%%%%%%%%%%%%%%%%%%%%%%%%%%%%%%%%%%%%%%%%%%%%%%%%%%%%%%%%%%%%%%%%%%%%%%%%%%%%%%%%%%%%
\section{Application}
Here we assess some test cases to show how the DAF method functions in practice. 

\subsection{One-Dimensional}

\subsection{Multi-Dimensional}
In a multi-dimensional application, we use the DAF to approximate a function such as
\begin{equation}
f(x,y,z) = \int\int\int f(x,y,z)\delta(x)\delta(y)\delta(z) {\rm d}x {\rm d}y {\rm d}z
\end{equation}
or in fector notation for simplicity: 
\begin{equation}
f({\bf r}) = \int_{\bf V} f({\bf r})\delta({\bf r}){\rm d}{\bf V}
\end{equation}
where $\delta({\bf r}) = \delta(r_1)\delta(r_2)\delta(r_3)...$ and ${\bf V}$ is the
multi-dimensional volume to which the DAF is being applied. 

In three spacial dimensions, these coordinates are typically x, y, and z, but the method is
not limited to strictly spacial coordinates, nor is it limited to the number of
dimensions. 



\section{ITEMS TO INCLUDE IN TEXT}
{\bf NOTE: Toeplitz Matrices?}

{\bf NOTE: The points at which values or derivatives of the function are wanted must be known
before construction of the DAF matrices. }

{\bf NOTE: Specialized DAF cases: periodic, symmetric. In multidimensional DAF applications, 
it might be good to label them by the irreducible representation.}

\appendix


\section{Proof of Lobatto Integration Formula}
\label{lobatappend}

%%%%%%%%%%%%%%%%%%%%%%%%%%%%%%%%%%%%%%%%%%%%%%%%%%%%%%%%%%%%%%%%%%%%%%%%%%%%%%%%%%%%%%%%%%
%%%%%%%%%%%%%%%%%%%%%%%%%%%%%%%%%%%%%%%%%%%%%%%%%%%%%%%%%%%%%%%%%%%%%%%%%%%%%%%%%%%%%%%%%%
\subsection{General Quadrature Proof}
\label{genproof}
A Gaussian-type quadrature formula for approximating an integral is based off of
a set of polynomials that are orthogonal with respect to a weighting function over
the range of integration. For the Legendre polynomials, the range of integration is 
$[-1,1]$ and the weight function is 1. 

For any given set of $n$ quadrature abscissas,$\{x\}_n$, weights can be calculated to make
the following summation exact, 
\begin{equation}
\intunit f(x)\dx = \sum_{i=1}^nw_if(x_i)
\end{equation}
for all $f$ if $f$ is a polynomial of order $n-1$ or less. An improvement can be made
by specifying a set of abscissas. In the example of Legendre polynomials (and this holds
true for other orthogonal polynomials and their related quadrature formulas), the set
of points is chosen to be the roots of the $P_n(x)$ Legendre polynomial. Using these
abscissas and then calculating the weights ensures a quadrature that will exactly 
integrate any polynomial of order $2n-1$. This can be shown as follows
\cite{Burden_gaussquad}, 

Say $g(x)$ is a polynomial of order no greater than $2n-1$. This polynomial can be
written as 
\begin{equation}
\label{genproof1}
g(x)=q(x)P_n(x)+r(x)
\end{equation}
where $q(x)$ is the dividend after dividing $g(x)$ by $P_n(x)$, and $r(x)$ is the 
remainder. If $g(x)$ is of order $2n-1$ or less, then both $q(x)$ and $r(x)$ cannot
be of order greater than $n-1$. As such, both $q(x)$ and $r(x)$ can be represented
as a linear combination of the set of Legendre polynomials of order $n-1$ and less. 
Integrating equation \ref{genproof1} gives, 
\begin{align}
\label{genproof2}
\intunit g(x)\dx &= \intunit q(x)P_n(x)\dx + \intunit r(x)\dx \\
\intunit g(x)\dx &= 0+\intunit r(x)\dx
\end{align}
where the first term on the right-hand side is necessarily zero because if $q(x)$ 
can be represented by an expansion in Legendre polynomials of orders less then $n$, 
then each of these polynomials are orthogonal to $P_n(x)$ with respect to integration.
Therefore, all that is left is the integration over the remainder term. 
In performing the same steps by quadrature, 
\begin{align}
\label{genproof3}
\sumunit g(x_i)w_i &= \sumunit q(x_i)P_n(x_i)w_i + \sumunit r(x_i)w_i \\
\sumunit g(x_i)w_i &= 0+\sumunit r(x_i)w_i
\end{align}
where again the first summation on the right-hand side is zero because the 
abscissa points were chosen to be the roots of $P_n(x)$. Because $r(x)$ is of
order $n-1$ or less, this term can be integrated exactly by quadrature, and thus
$g(x)$ is integrated exactly. Thus for any polynomial $g(x)$ of order $2n-1$, 
the $n$ point quadrature can be made to be exact, giving
\begin{equation}
\intunit g(x)\dx = \sumunit g(x_i)w_i
\end{equation}

%%%%%%%%%%%%%%%%%%%%%%%%%%%%%%%%%%%%%%%%%%%%%%%%%%%%%%%%%%%%%%%%%%%%%%%%%%%%%%%%%%%%%%%%%%
\subsection{Lobatto Quadrature Proof}
\label{lobattoproof}
The Lobatto integration formula is 
\begin{equation}
\label{lobatint_proof}
\intunit f(x) \approx \frac{2}{n(n-1)}\left[f(-1)+f(1)\right]+\sum_{i=2}^{n-1}w_if(x_i)
\end{equation}
where the abscissas, $x_i$ are the $i^{th}$ root of $P_{n-1}^\prime(x)$, $P_n(x)$ being the 
$n^{th}$ Legendre polynomial, and the interior weights, $w_i$ are given by the formula
\begin{equation}
\label{weightform}
w_i=\frac{2}{n(n-1)[P_{n-1}(x)]^2}
\end{equation}
This integration quadrature is exact for all functions $f(x)$ such that $f(x)$ is a
polynomial of order $2n-3$ or less. The proof of this relies on the Legendre Polynomial
recursion relation \cite{LegendrePoly},
\begin{equation}
\label{P3term}
(n+1)P_{n+1}(x)=(2n+1)xP_n(x)-nP_{n-1}(x) 
\end{equation}
and the Legendre polynomial derivative formula \cite{LegendrePoly}
\begin{align}
\label{P3der}
(1-x^2)P^\prime_{n}(x) &= -nxP_n(x)+nP_{n-1}(x) \\
  {}  &=(n+1)\left[xP_n(x)-P_{n+1}(x)\right]
\end{align}
The main difference between the Lobatto Quadrature formula and other Gaussian-type
quadratures is that the end points of the integration range ($[-1,1]$, generally) are
included as quadrature points. By including these points explicitly, two degrees 
of freedom are lost, and thus the quadrature is not as accurate as the general case. 

By including the end points, the abscissas for Lobatto quadrature are the roots
of the equation
\begin{equation}
\label{Phidef1}
\Phi_n(x)=(1-x^2)P_{n-1}^\prime(x)
\end{equation}
That these abscissas are those that provide the maximum accuracy is proven 
in a subsequent section. 

The function $\Phi_n(x)$ is a polynomial of order $n$ and has roots at $x=-1,1$. 
From the relations given in equations \ref{P3term} and \ref{P3der}, equation \ref{Phidef1}
can be rewritten, 
\begin{align}
\label{Phidef2}
\Phi_n(x)&=(1-x^2)P_{n-1}^\prime(x) \nonumber \\
&=(n-1)P_{n-2}(x)-(n-1)xP_{n-1}(x) \nonumber \\
&=(n-1)P_{n-2}(x)-\frac{n-1}{2n-1}\left(nP_n(x)+(n-1)P_{n-2}(x)\right) \nonumber \\
&=\frac{n(n-1)}{2n-1}\left(P_{n-2}(x)-P_n(x)\right)
\end{align}
From here the proof proceeds in the same way as in the general case. 
Given a polynomial, $g(x)$ that is of order $2n-3$ or less, it can be written 
similarly to equation \ref{genproof1}, but now we use the $\Phi$ function instead of 
a Legendre Polynomial as the divisor, 
\begin{align}
\label{gproof1}
g(x)&=q(x)\Phi_n(x)+r(x) \nonumber \\
g(x)&=q(x)\frac{n(n-1)}{2n-1}\left(P_{n-2}(x)-P_n(x)\right) +r(x) 
\end{align}
If $g(x)$ is of order $2n-3$ or less, then $q(x)$ is of order $n-3$ or less. The remainder
$r(x)$ is of order $n-1$ or less, as the highest order possible of the remainder is always one
less than then divisor. Preceding as before by integrating both sides gives
\begin{equation}
\label{gproof2}
\intunit g(x)\dx = \intunit q(x)\frac{n(n-1)}{2n-1}\left(P_{n-2}(x)-P_n(x)\right)\dx + \intunit r(x)\dx 
\end{equation}
In general, the first term on the right-hand side is only zero if $q(x)$ can be 
expanded in Legendre Polynomials of order $n-3$ or less, due to the presence of the
$P_{n-2}(x)$ polynomial. It is possible that $q(x)$ have order $n-1$, but only if 
$q(x)$ and $P_{n-2}(x)$ remain orthogonal, and this is generally not the case. 
After integration, we again have the condition
\begin{equation}
\label{gproof3}
\intunit g(x)\dx = \intunit r(x)\dx 
\end{equation}

Performing the Lobatto quadrature on equation \ref{gproof1} behaves similarly to the
general quadrature case. 
\begin{align}
\label{gproof4}
\sumunit g(x_i)w_i &= \sumunit q(x_i)w_i\frac{n(n-1)}{2n-1}\left(P_{n-2}(x_i)-P_n(x_i)\right) + \sumunit r(x_i)w_i \nonumber \\
\sumunit g(x_i)w_i &= 0 + \sumunit r(x_i)w_i
\end{align}
where again the first term on the right-hand side is zero because the quadrature
abscissas were chosen to be roots of $\Phi(x)$ (as defined in equation \ref{Phidef2}). 

Therefore it is proved that the $n$-point Lobatto Quadrature is exact for all 
polynomials of order $2n-3$. 

%%%%%%%%%%%%%%%%%%%%%%%%%%%%%%%%%%%%%%%%%%%%%%%%%%%%%%%%%%%%%%%%%%%%%%%%%%%%%%%%%%%%%%%%%%
\subsection{Proof of Weight Formula for Lobatto Quadrature}
Because the Lobatto integration formula exactly integrates any polynomial of order
$2n-3$ or less, integrating a Lagrange interpolating polynomial can be done by 
quadrature exactly. A Lagrange interpolating polynomial is defined in equation 
\ref{lipdef} and rewritten here for convenience, 
\begin{equation}
\lambda_k(x)=\prod_{j \ne k}^n\frac{(x-x_j)}{(x_k-x_j)} \nonumber
\end{equation}
where $\lambda_k(x_k)=1$ by definition and is zero at all other quadrature abscissas. 
Integrating this function produces the quadrature result, 
\begin{align}
\label{lipint1}
\intunit \lambda_k(x)\dx &= \sumunit \lambda_k(x_i)w_i \nonumber \\
&= w_k\lambda_k(x_k) = w_k
\end{align}
Therefore a formula for the integration of $\lambda_k(x)$ will result in 
an expression for the weight factors. 

The Lagrange interpolating polynomial can also be written in another form
based on the Legendre Polynomials. Since the $\lambda_k(x)$ functions share
all the same roots as the expression $(1-x*2)P^\prime_{n-1}(x)$ except for
the $k^{th}$ root, $\lambda_k(x)$ can be defined as
\begin{equation}
\lambda_k(x)=Q\frac{(1-x^2)P^\prime_{n-1}(x)}{(x-x_k)}
\end{equation}
where $Q$ is a normalizing factor that ensures $\lambda_k(x_k)=1$. 
The normalization factor $Q$ can be calculated by use of l'Hopital's rule
applied to $\lambda_k(x)$ or more simply by the identity \cite{Dahlquist_gaussquad}
\begin{equation}
\label{lipdef2}
\lambda_k(x)=\frac{\Phi(x)}{(x-x_k)\Phi^\prime(x_k)}
\end{equation}
where $\Phi(x)=(1-x^2)P_{n-1}^\prime(x)$ and is also defined in equations \ref{Phidef1}
and \ref{Phidef2}. Equation \ref{lipdef2} is easy to prove given that all terms in 
$\Phi^\prime(x)$ that contain the $(x-x_k)$ factor after differentiation are zero when
$\Phi^\prime(x)$ is evaluated at $x=x_k$. 
The result is that with normalizing, 
\begin{equation}
\label{lipdef3}
\lambda_k(x)=\frac{(1-x^2)P^\prime_{n-1}(x)}{(x-x_k)(1-x_k^2)P^{\prime\prime}(x_k)}
\end{equation}

It is now left to prove that integrating this expression for $\lambda_k(x)$ gives
the weight formula defined in equation \ref{weightform}
\begin{align}
\intunit \lambda_k(x)\dx & = \intunit \frac{(1-x^2)P^\prime_{n-1}(x)}{(x-x_k)(1-x_k^2)P_{n-1}^{\prime\prime}(x_k)} \nonumber \\
 & = \frac{1}{(1-x_k^2)P^{\prime\prime}_{n-1}(x_k)} \intunit \frac{(1-x^2)P^\prime_{n-1}(x)}{(x-x_k)} \nonumber \\
\label{wform1}
 & = \frac{2}{n(n-1)[P_{n-1}(x_k)]^2} \\
\label{wform2}
 & = \frac{2n}{(1-x_k^2)P^{\prime\prime}_{n-1}(x_k)P^\prime_n(x_k)} 
\end{align}
Where the forms for the weights in equations \ref{wform1}  and \ref{wform2} are given in references
\cite{ABRW-STGN} and \cite{LobattoQuad} respectively. 
From equation \ref{wform2} we can see the proof of the weights reduces to proving the equality
\begin{equation}
\label{weightint}
\intunit \frac{(1-x^2)P^\prime_{n-1}(x)}{(x-x_k)}\dx = \frac{2n}{P^\prime_n(x_k)}
\end{equation}

From here we shall prove that the interior abscissas are indeed the roots of $P_{n-1}^\prime(x)$
and show how to perform the integral in equation \ref{weightint} to get the formulae for 
the associated weights.

%%%%%%%%%%%%%%%%%%%%%%%%%%%%%%%%%%%%%%%%%%%%%%%%%%%%%%%%%%%%%%%%%%%%%%%%%%%%%%%%%%%%%%%%%%
\subsubsection{Abscissas}
As noted before, the interior weights can be calculated from the solution
to the integral
\begin{equation}
\intunit \lambda_i(x)\dx = w_i
\end{equation}
where $\lambda_i(x)$ is the Lagrange interpolating polynomial defined by the 
chosen abscissas. For the Lobatto quadrature, the interior abscissas are defined
to be the roots of $P^\prime_{n-1}(x)$, where $P_n(x)$ is the $n^{th}$ Legendre 
polynomial and $n$ is the total number of quadrature points (including the end points). 

This section proves that the best possible abscissas are the roots of $P^\prime_{n-1}(x)$
and proves the formula, equation \ref{weightform}, for the weights. 

First let us presume that we do not know the interior abscissas, but that the
end point abscissas have been forced. Earlier in equation \ref{Phidef1}, $\Phi$
was defined as the polynomial with roots at all of the abscissas. Now let us 
redefine this function as one in which the interior abscissas are unknown, 
\begin{equation}
\label{Phiudef}
\Phi_n(x)=(1-x^2)\phi_{n-2}(x)
\end{equation}
where $\phi_{n-2}(x)$ is some unknown function with $n-2$ real, distinct roots
in the range of $[-1,1]$. Here we will follow the formulation of Hildebrand 
\cite{Hildebrand}, chapters 7-8, on quadratures with assigned abscissas. 
By the methods of section \ref{genproof} we know that maximum accuracy of the 
quadrature is obtained by using the roots of an orthogonal polynomial. We 
therefore make the assertation that $\phi_{n-2}$ is the $(n-2)^{th}$ member of
a set of orthogonal polynomials that are orthogonal with respect to the 
weighting function $\wbar \equiv (1-x^2)$. Based on this assertation, we must
ensure that $\phi_{n-2}$ is orthogonal to all polynomials of degree $n-3$ or less. 
For simplicity in notation, from here on we will set $r=n-2$. 
Thus, 
\begin{equation}
\label{phiortho}
\intunit \wbar(x)\phi_r(x)g_{r-1}(x)\dx=0
\end{equation}
where $g_{r-1}(x)$ is any general polynomial of degree $r-1$ or less. 
Following section 7.5 of Hildebrand\cite{Hildebrand}, we proceed to reform
equation \ref{phiortho} through successive integration-by-parts. In doing
so we require the $r^{th}$ integral of $\wbar(x)\phi_r(x)$, and so we define
this to be the $r^{th}$ derivative of a function $U_r(x)$, 
\begin{equation}
\label{Udef}
\wbar(x)\phi_r(x)=\frac{d^rU_r(x)}{dx^r} = U^{(r)}_r(x)
\end{equation}
and from this equation \ref{phiortho} is now
\begin{equation}
\label{Uortho}
\intunit U^{(r)}_r(x)g_{r-1}(x)\dx = 0
\end{equation}
After $r$ derivatives, the $g_{r-1}(x)$ function is zero and the integral term
vanishes. What is left is $r-1$ surface terms given as
\begin{align}
\label{Usurface}
\left\{ U_r^{(r-1)}(x)g_{r-1}^{(1)}(x)\right. &- U_r^{(r-2)}(x)g_{r-1}^{(2)}(x) + U_r^{(r-3)}(x)g_{r-1}^{(3)}(x) + \dots \nonumber \\
 \dots &+ \left. \left. (-1)^{r-1}U_r(x)g_{r-1}^{(r-1)}(x)\right\} \right|_{-1}^{+1}  = 0
\end{align}
That the sum of the surface terms is zero is necessary to ensure that the integral
in equation \ref{Uortho} is  zero. Because $g_{r-1}(x)$ can be {\it any} 
polynomial of degree $r-1$ or less, there are no constraints on its value at 
$x=\pm1$. Therefore, we have the constraint on $U_r(x)$ that 
\begin{equation}
\label{Uend}
U_r^{(r-k)}(-1)=U_r^{(r-k)}(1)=0, \quad\quad k=1,2,\dots,r
\end{equation}

Furthermore, we have the requirement that if $\phi_r(x)$ is a polynomial of
degree $r$, its $(r+1)^{th}$ derivative must vanish. From this requirement 
and equation \ref{Udef}, $U_r(x)$ must be a solution to the differential equation
\begin{equation}
\label{Udifeq}
\frac{d^{r+1}}{dx^{r+1}}\left[\frac{1}{\wbar(x)}\frac{d^rU_r(x)}{dx^r}\right]=
\frac{d^{r+1}}{dx^{r+1}}\left[\frac{1}{(1-x^2)}\frac{d^rU_r(x)}{dx^r}\right]=0
\end{equation}
where $U_r(x)$ must satisfy the conditions of equation \ref{Uend}. 

From looking at equation \ref{Udifeq}, we can see that the $1/(1-x^2)$ term may
be problematic; no number of successive derivatives cause this term to vanish. 
Therefore, this term must be cancled from the $U_r^{(r)}(x)$ term. Still requiring 
that equation \ref{Udifeq} be zero then allows $U_r(x)$ to be a polynomial of order
$2r+2$. The requirement of equation \ref{Usurface} leads us to try a function for $U_r(x)$
of the form
\begin{equation}
\label{U1}
U_r(x)=C_{r,1}\frac{d^2}{dx^2}(x^2-1)^{r+2}-C_{r,2}(x^2-1)^r
\end{equation}
Where $C_{r,1}$ and $C_{r,2}$ is a constant coefficient. 
Clearly this function has an order of $2r+2$, and though it may seem strange to chose 
this function, the reasons will be made clear shortly. We are free to choose these
coefficients provided that the roots of $\phi_r$ remain the same and the requirement
of equation \ref{Usurface} is maintained. The use of the $(x^2-1)$ terms ensure that 
the conditions of equation \ref{Usurface} are met; each term in the first $r$ derivatives
of this definition of $U_r(x)$ contain at least on $(x^2-1)$ term and are thus zero at
$x=\pm1$.  After $r$ applications of the derivative operator with respect to $x$ and
division by $\wbar(x)$, 
$U_r^{(r)}(x)$ is a polynomial of order $r=n-2$, and thus also satisfies the condition 
of equation \ref{Udifeq}, which is to say the $(r+1)^{th}$ derivative of this polynomial
is zero. Hence, all conditions on the form of $U_r(x)$ are met. 

In order to find $\phi_r(x)$, we must work out the explicit form of $U_r^{(r)}(x)$. 
We first apply the derivative operators, and since the conditions from the surface terms
are met, those of equation \ref{Usurface}, by the $(x^2-1)$ terms, we are free to 
manipulate the coefficients later. Applying $d^r/dx^r$ to equation \ref{U1}, we have
\begin{equation}
\label{U2}
\frac{d^r}{dx^r}U_r(x)=C_{r,1}\frac{d^{r+2}}{dx^{r+2}}(x^2-1)^{r+2}-C_{r,2}\frac{d^r}{dx^r}(x^2-1)^r
\end{equation}
Observing that there is a similarity in this terms to the Rodrigues formula for the 
Legendre polynomials, defined as
\begin{equation}
\label{RodriguesLegendre}
P_l(x)=\frac{1}{2^ll!}\frac{d^l}{dx^l}(x^2-1)^l
\end{equation}
We can simplify this equation by the choice of the coefficients to be
\begin{align}
\label{Crcoef1}
C_{r,1}&=-C\frac{(r+1)(r+2)}{2r+1}\frac{1}{2^{r+2}(r+2)!} \\
\label{Crcoef2}
C_{r,2}&=C\frac{(r+1)(r+2)}{2r+1}\frac{1}{2^{r}r!} 
\end{align}
and rewrite equation \ref{U2} as
\begin{align}
\frac{d^r}{dx^r}U_r(x) &= -C\frac{(r+1)(r+2)}{2r+1}\left[ \frac{1}{2^{r+2}(r+2)!}\frac{d^{r+2}}{dx^{r+2}}(x^2-1)^{r+2}-\frac{1}{2^rr!}\frac{d^r}{dx^r}(x^2-1)^r\right] \\
\frac{d^r}{dx^r}U_r(x)&=-C \frac{(r+1)(r+2)}{2r+1}\left[ P_{r+2}(x)-P_r(x) \right] 
\end{align}
again with $r=n-2$, then we have the following algebraic sequence, 
\begin{align}
\frac{d^r}{dx^r}U_r(x) &= -\frac{C(r+1)(r+2)}{2r+1}\left[ P_{r+2}(x)-P_r(x) \right]  \\
\frac{d^{n-2}}{dx^{n-2}}U_{n-2}(x) &= -\frac{Cn(n-1)}{2n-1}\left[ P_{n}(x)-P_{n-2}(x) \right]  \\
 &= C\frac{-n^2+n}{2n-1}\left[ P_{n}(x)-P_{n-2}(x) \right]  \\
 &= C\frac{n^2-n(2n-1)}{2n-1}P_{n}(x)-\frac{n(n-1)}{2n-1}P_{n-2}(x)   \\
\label{ptmp1}
 &= C\frac{n^2}{2n-1}P_n(x)-nP_{n}(x)-\frac{n(n-1)}{2n-1}P_{n-2}(x)   \\
\label{ptmp2}
 &= C\left(nxP_{n-1}(x)-nP_{n}(x)\right)\\
\label{ptmp3}
 &=C(1-x^2)P_{n-1}^\prime(x)
\end{align}
where the step between lines \ref{ptmp1} and \ref{ptmp2} utilizes the three-term
recurrance relationship of 
\begin{equation}
nxP_{n-1}(x)=\frac{n^2}{2n-1}P_n(x)+\frac{n(n-1)}{2n-1}P_{n-2}(x)
\end{equation}
and between lines \ref{ptmp2} and \ref{ptmp3} the Legendre derivative 
equation is used, 
\begin{equation}
(1-x^2)P_{n-1}^\prime(x)=nxP_{n-1}(x)+nP_{n}(x)
\end{equation}

Placing $U_r^{(r)}(x)=C(1-x^2)P_{r+1}^\prime(x)=(1-x^2)P_{n-1}^\prime(x)$ into 
equation \ref{Udifeq} gives
\begin{equation}
\frac{d^{n-1}}{dx^{n-1}}\left[ \frac{1}{(1-x^2)}C(1-x^2)P_{n-1}^\prime(x) \right]=0
\end{equation}
which is true, being that $P_{n-1}^\prime(x)$ is a polynomial of order $n-2$, 
the $(n-1)^{th}$ derivative of it must vanish. Therefore, the polynomial that
we are seeking, $\phi_{n-2}(x)$, is equal or proportional to $P_{n-1}^\prime(x)$. 
The coefficient $C$ is determined such that the lowest order
polynomial of this set, $\phi_0(x)=CP_1^\prime(x)=1$ which give $C=1$. 
Therefore, $\phi_{n-2}(x)=P_{n-1}^\prime(x)$, and  is a member of a set that are 
orthogonal with respect to the weighting function $(1-x^2)$. 

This proves that for an $n$ point quadrature including the end points, the interior
abscissas are in fact the roots of $P_{n-1}^\prime(x)$. 
%%%%%%%%%%%%%%%%%%%%%%%%%%%%%%%%%%%%%%%%%%%%%%%%%%%%%%%%%%%%%%%%%%%%%%%%%%%%%%%%%%%%%%%%%%

\subsubsection{End Point Weights}
Following is the proof for the end-point weights for the Lobatto Quadrature. From 
equation \ref{lobatint_proof} we can see that at the end points of $x=\pm 1$ the 
quadrature weight is $2/(n(n-1))$. To prove this we need to evaluate the integral of 
equation \ref{weightint} for the abscissas $x=\pm1$. For $x=1$, this is
\begin{equation}
\label{w_xp_1}
\intunit \frac{\beta(1-x^2)P^\prime_{n-1}(x)}{(1-x)}\dx = w_{x=1}
\end{equation}
where $\beta$ is the normalization factor to ensure that the polynomial is equal 
to $1$ at $x=1$. Because the numerator factors directly, l'Hopital's rule is unnecessary
to find the normalization, and the integral can be written simply as
\begin{equation}
\label{w_xp_2}
\intunit \frac{(1+x)P^\prime_{n-1}(x)}{2P^\prime_{n-1}(1)}\dx = w_{x=1}
\end{equation}

First we must evaluate the expression $2P^\prime_{n-1}(1)$. The first several values
of $P^\prime_m(x=1)$ are
\begin{align}
\label{Ppat1}
P^\prime_0(1) &=0 \nonumber \\
P^\prime_1(1) &=1 \nonumber \\
P^\prime_2(1) &=3 \nonumber \\
P^\prime_3(1) &=6 \nonumber \\
P^\prime_4(1) &=10
\end{align}
and so on. From observation we can see there is a formula of 
\begin{equation}
\label{Ppat1_2}
P^\prime_n(1)=\sum_i^ni = \frac{n(n+1)}{2}
\end{equation}
This relation can also be proved from the recursion relation and the definition
of the Legendre polynomials. Using equation \ref{Ppat1_2} to rewrite equation 
\ref{w_xp_2} gives
\begin{equation}
\label{w_xp_3}
\intunit \frac{(1+x)P^\prime_{n-1}(x)}{n(n-1))}\dx = w_{x=1}
\end{equation}
This integral can now be evaluated by integration-by-parts, 
\begin{align}
n(n-1)w_{x=1} &= \intunit (1+x)P^\prime_{n-1}(x)\dx \nonumber\\
&= \intunit P^\prime_{n-1}(x)\dx + \intunit xP^\prime_{n-1}(x)\dx \nonumber \\
&= P_{n-1}(x)\vert^1_{-1} + xP_{n-1}(x) \vert^1_{-1} - \intunit P_{n-1}(x)\dx \nonumber \\
&= P_{n-1}(1)-P_{n-1}(-1)+P_{n-1}(1)+P_{n-1}(-1) - 0 \nonumber \\
&= P_{n-1}(1)+P_{n-1}(1)\nonumber \\
\label{xp_int}
&=2
\end{align}
Note that $P_l(1)=1$ for all $l$, and the integral over any Legendre polynomial
other than $P_0(x)$ is zero. Since $n$ represents the number of points in the integration,
a one point quadrature is meaningless, since at the very least we count the end points.  
Therefore, we can rule out the integral over $P_{n-1}$ 
surviving. Thus for $w_{x=1}$ we have
\begin{equation}
w_{x=1}=\frac{2}{n(n-1)}
\end{equation}
just as in equation \ref{lobatint_proof}. 

For the weight at $x=-1$, the procedure is very similar, but because the Legendre
polynomials are alternatively even and odd, their values at $x=-1$ alternate 
between $\pm1$ and the derivatives, while still the same magnitude as those
at $x=1$, also alternate between positive and negative.  
The formula for the derivative value at $x=-1$ is 
\begin{equation}
\label{w_xn_1}
P^\prime_{n-1}=\frac{1}{2}(-1)^nn(n-1)
\end{equation}
The integral to be solved is
\begin{equation}
\label{w_xn_2}
\intunit \frac{(-1)^n(1-x)P^\prime_{n-1}(x)}{n(n-1))}\dx = w_{x=1}
\end{equation}
and proceding as before in equation \ref{xp_int}
\begin{align}
n(n-1)w_{x=-1} &= \intunit (1-x)P^\prime_{n-1}(x)\dx \nonumber \\
&= \intunit P^\prime_{n-1}(x)\dx - \intunit xP^\prime_{n-1}(x)\dx \nonumber \\
&= P_{n-1}(x)\vert^1_{-1} - xP_{n-1}(x) \vert^1_{-1} + \intunit P_{n-1}(x)\dx \nonumber \\
&= P_{n-1}(1)-P_{n-1}(-1)+P_{n-1}(1)+P_{n-1}(-1) - 0 \nonumber \\
&= P_{n-1}(-1)+P_{n-1}(-1) = 2(-1)^n
\label{xn_int}
\end{align}
which gives
\begin{equation}
w_{x=-1}=\frac{(-1)^n2}{(-1)^nn(n-1)} = \frac{2}{n(n-1)}
\end{equation}
and so the values of the weights at $x=\pm1$ are proved.

%%%%%%%%%%%%%%%%%%%%%%%%%%%%%%%%%%%%%%%%%%%%%%%%%%%%%%%%%%%%%%%%%%%%%%%%%%%%%%%%%%%%%%%%%%

\subsubsection{Interior Weights}
Here we wish to prove that the values for the interior integration weights is given 
by the formulae, 
\begin{equation}
\label{weightform2}
w_i=\frac{2}{n(n-1)[P_{n-1}(x)]^2} = -\frac{2n}{(1-x_i^2)P_{n-1}^{\prime\prime}(x_i)P_n^\prime(x_i)}
\end{equation}

Beginning with equation \ref{weightint} and the knowledge that $P_{n-1}^\prime(x)$ is
the $(n-2)^{th}$ member of a set of polynomials orthogonal with respect to $(1-x^2)$, 
we can now exactly calculate the integral, 
\begin{equation}
\label{pint}
\intunit \frac{(1-x^2)P^\prime_{n-1}(x)}{(x-x_i)}\dx = \frac{2n}{P^\prime_n(x_i)} \nonumber
\end{equation}
where $x_i$ is the $i^{th}$ root of $P^\prime_{n-1}(x)$.

Keeping the simpler notation of $\phi_r(x)=P_{n-1}^\prime(x)$ with $r=n-2$, we make
use of the Christoffel-Darboux identity \cite{Hildebrand}, 
\begin{equation}
\label{CDident}
\sum_{k=0}^{m}\frac{\phi_k(x)\phi_k(y)}{\gamma_k}=\frac{\phi_{m+1}(x)\phi_m(y)-\phi_m(x)\phi_{m+1}(y)}{a_m\gamma_m(x-y)}
\end{equation}
where $a_m=A_{m+1}/A_m$ with $A_m$ is the coeffcient of the $x^m$ term of $\phi_m(x)$, and $\gamma_m$ 
is defined as \
\begin{equation}
\label{gammadef}
\gamma_m=\intunit w(x)\left(\phi_m(x)\right)^2\dx
\end{equation}

Now we define $y=x_i$, where $x_i$ is the $i^{th}$ root of the $\phi_m(x)$ polynomial. 
With this definition, equation \ref{CDident} becomes, 
\begin{equation}
\label{CDreduced}
\sum_{k=0}^{m}\frac{\phi_k(x)\phi_k(x_i)}{\gamma_k}=\frac{-\phi_m(x)\phi_{m+1}(x_i)}{a_m\gamma_m(x-x_i)}
\end{equation}

Multiplying both sides of equation \ref{CDreduced} by $(1-x^2)\phi_0(x)$ and integrating gives
\begin{align}
\intunit (1-x^2)\phi_0(x)\sum_{k=0}^{m}\frac{\phi_k(x)\phi_k(x_i)}{\gamma_k}\dx &= \intunit (1-x^2)\phi_0(x)\frac{-\phi_m(x)\phi_{m+1}(x_i)}{a_m\gamma_m(x-x_i)}\dx \nonumber \\
\sum_{k=0}^{m}\frac{\phi_k(x_i)}{\gamma_k}\intunit (1-x^2)\phi_0(x)\phi_k(x)\dx &= \intunit (1-x^2)\phi_0(x)\frac{-\phi_m(x)\phi_{m+1}(x_i)}{a_m\gamma_m(x-x_i)}\dx \nonumber \\
\frac{\phi_0(x_i)}{\gamma_0}\gamma_0 &= \intunit \frac{-(1-x^2)\phi_0(x)\phi_m(x)\phi_{m+1}(x_i)}{a_m\gamma_m(x-x_i)}\dx \nonumber \\
\phi_0(x_i)=1 &= \frac{\phi_{m+1}(x_i)}{a_m\gamma_m}\intunit (1-x^2)\phi_0(x)\frac{-\phi_m(x)}{(x-x_i)}\dx \nonumber \\
\label{CDseq}
\frac{-a_m\gamma_m}{\phi_{m+1}(x_i)} &= \intunit (1-x^2)\phi_0(x)\frac{\phi_m(x)}{(x-x_i)}\dx 
\end{align}
Now changing back to Legendre polynomials, we have
\begin{equation}
\label{CDPend}
\intunit \frac{(1-x^2)P_{n-1}^\prime(x)}{x-x_i}\dx = \frac{-a_{n-1}\gamma_{n-1}}{P_n^\prime(x_i)}
\end{equation}

The leading coefficients of the Legendre polynomials is given by $A_n=(2n)!/[2^n(n!)^2]$, and 
therefore the leading coefficients of $P_n^\prime(x)$, $A^\prime_n$, are given by  
\begin{align}
\label{Aprime}
A^\prime_n&=\frac{n(2n)!}{2^n(n!)^2} \\
a_{n-1}&=\frac{A^\prime_n}{A^\prime_{n-1}}\\
&=\frac{n(2n)!2^{n-1}((n-1)!)^2}{(n-1)(2n-2)!2^n(n!)^2} \\
&= \frac{2n-1}{n-1}
\end{align}
It can also be shown that 
\begin{equation}
\label{gammasolve}
\gamma_{n-1}=\intunit(1-x^2)\left(P_{n-1}^\prime(x)\right)^2 \dx = \frac{2n(n-1)}{2n-1}
\end{equation}
Employing these values for $a_{n-1}$ and $\gamma_{n-1}$ in equation \ref{CDPend}
we have 
\begin{equation}
\intunit \frac{(1-x^2)P_{n-1}^\prime(x)}{x-x_i}\dx = \frac{2n}{P_n^\prime(x_i)}
\end{equation}
which is precisely equation \ref{pint}, which we initially set out to prove, and we 
are left with the equation 
\begin{equation}
\label{weightform3}
w_i= -\frac{2n}{(1-x_i^2)P_{n-1}^{\prime\prime}(x_i)P_n^\prime(x_i)}
\end{equation}
for the interior integration weights. 

This concludes the derivation of the Lobatto Quadrature. 

\section{Integration Methods for the DAF Operators}

A crucial step in the DAF process is the accurate numerical evaluation of the
convolution integrals. In the case where the functions in question have 
compact support, that is, they go to zero at $x\to\pm\infty$. 






\bibliographystyle{apsrev}
\bibliography{biblist}

\end{document}
