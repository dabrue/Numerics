\documentclass[preprint]{revtex4}

%%%%%%%%%%%%%%%%%%%%%%%%%%%%%%%%%%%%%%%%%%%%%%%%%%%%%%%%%%%%%%%%%%%%%%%%%%%%%%%%%%%%%%%%%%
\usepackage{amsmath}
\newcommand{\rtpi}{\ensuremath{\sqrt{\pi}}}
\newcommand{\intl}{\int\displaylimits}
\newcommand{\intunit}{\ensuremath{\int_{-1}^{1}}}
\newcommand{\sumunit}{\ensuremath{\sum_{i=1}^{n}}}
\newcommand{\dx}{{\rm d}x}
\newcommand{\dy}{{\rm d}y}
\newcommand{\wbar}{\ensuremath{\bar{w}}}

%%%%%%%%%%%%%%%%%%%%%%%%%%%%%%%%%%%%%%%%%%%%%%%%%%%%%%%%%%%%%%%%%%%%%%%%%%%%%%%%%%%%%%%%%%
\begin{document}

\title{%
    Distributed Approximating Functionals\\
    %\vspace{0.5cm}
    {\small Theory and Computational Methodology}\\
}
\author{Daniel A. Brue}
%\email{danielbrue (at) Gmail}
\date{\today}

\begin{abstract}
The abstract goes here after the paper is done.
\end{abstract}

\maketitle

%%%%%%%%%%%%%%%%%%%%%%%%%%%%%%%%%%%%%%%%%%%%%%%%%%%%%%%%%%%%%%%%%%%%%%%%%%%%%%%%%%%%%%%%%%
\section{Introduction}
The distributed approximating functionals (DAF) method was developed 
and implemented for solving classes of differential equations very accurately
with a high fidelity to the accuracy of derivatives. 
The DAF method was applied first to time-dependent propgation problems in theoretical
quantum physics applications\cite{Kouri-DAF1,Kouri-DAF2,Kouri-DAF3,Kouri-DAF4,
Kouri-DAF5,Kouri-DAF6,Kouri-DAF7}.

The DAF method has been used in various situations that involve time evolution, 
from quantum particle trajectories\cite{QTM}, integration of Feynman path integrals\cite{DAF-Szalay1},
and fluid dynamics\cite{NVS,Burgers}, and also as a method of solving bound eigenvalue-eigenvector
problems\cite{DAF-Szalay2,SGWD} and data interpolation\cite{DAF-Szalay3}.

The DAF method is more computationally intensive in practice than other simpler methods, 
however, much can be precomputed and reused, so the most time consuming aspects of the
method need be computed only once 

\begin{enumerate}
\item{\bf Differential Equations}: The DAF method constructs matrix operators that
	can be applied to multidimentional linear differential equations. 
\item{\bf Interpolation}: The DAF method can be used to approximate functions and
	their derivatives with high accuracy. 
\end{enumerate}

%-----------------------------------------------------------------------------------------

%-----------------------------------------------------------------------------------------
\section{Mathematical Backgound}
In this section, we develop the mathematical foundation for the DAF method, establish 
the needed relations to employ the DAF in practice, and define its limitations. 

\subsection{Hermite Expansion of Delta Functions}
The DAF method uses an approximation of the Dirac delta function generated by 
an expansion in Hermite polynomials. With the appropriate Gaussian weight function, 
the Hermite polynomials form a complete and orthogonal basis on the domain of 
$(-\infty,\infty)$. 
This basis set can be used to expand any function on this domain. The advantages
of this method include high degrees of accuracy in representing other functions, 
and provides an analytic, functional approximation of undefined functions such as 
what might be represented by discrete data points. 

The expansion follows the general formula\cite{HermiteWiki} of 
\begin{equation}
\label{eqn_deltaHexp0}
\delta(x) = \sum_{m=0}^\infty h_m H_m\left(\frac{x}{\sigma}\right) e^{-\left(\frac{x}{\sigma}\right)}
\end{equation}
Where $H_m$ are the Hermite polynomials, $\sigma$ is a scaling function, and $h_m$ are the
expansion coefficients. A few notes on these terms: 
\begin{itemize}
\item{$H_m(x)$}: There are two common definitions for the Hermite polynomials, which are referred
to commonly as the {\it probability} and the {\it physics} definitions, which here we follow
Wikipedia's notation and use $H_m$ for the physics definition and ${\it He}_m$ for the probabilistic. 
They are related by ${\it He}_m = 2^m H_m$. In this document, we use the physics definition. 
\item{$\sigma$}: The scaling function $\sigma$ can be used to scale the Hermite functions 
``horizontally''. In the practical case in which the infinite sum in equation \ref{eqn_deltaHexp0}
is limited to a finite number of terms, the $\sigma$ factor directly effects how many terms
are needed to represent a function. A poorly chosen value of $\sigma$ 
\item{$h_m$}: The expansion coefficients represent how much each Hermite term contributes to the
expansion. These can be derived analytically for the $\delta(x)$ expansion.
This derivation is shown below. 
\end{itemize}

To determine the expansion coefficients $h_m$, we begin with equation \ref{eqn_deltaHexp0}
and multiply each side by $H_n(x/\sigma)$ and integrate over $(-\infty,\infty)$, 
\begin{eqnarray}
\label{eqn_deltaHexp1}
\intl_{-\infty}^{\infty}H_n\left(\frac{x}{\sigma}\right)\delta(x)dx & = & \intl_{-\infty}^{\infty}H_n\left(\frac{x}{\sigma}\right)\sum_{m=0}^\infty h_m H_m\left(\frac{x}{\sigma}\right) e^{-\left(\frac{x}{\sigma}\right)} dx \\*
H_n(0) & = & 
\sum_{m=0}^\infty h_m \intl_{-\infty}^{\infty}H_n\left(\frac{x}{\sigma}\right)
H_m\left(\frac{x}{\sigma}\right) e^{-\left(\frac{x}{\sigma}\right)} dx 
\label{eqn_deltaHexp2}
\end{eqnarray}
Next we use the orthogonality property\cite{HermiteWiki,HermitePolyMW} of 
\begin{equation}
\label{eqn_Hortho}
\intl_{-\infty}^{\infty}H_m(x)H_n(x)e^{-x^2}dx = \rtpi\, 2^m m!\, \delta_{mn}
\end{equation}
where $\delta_{mn}$ is the Kronecker Delta function and is equal to $1$ if $m=n$ and $0$ otherwise. 
With a scaling factor, $\sigma$, this becomes
\begin{equation}
\label{eqn_Hortho1}
%\intl_{-\infty}^{\infty}H_m\left(\frac{x}{\sigma}x\right)H_n\left(\frac{x}{\sigma}\right)e^{-\left(\frac{x}{\sigma}x\right)^2dx = \rtpi 2^m m! \delta_{mn}
\intl_{-\infty}^{\infty} H_m\left(\frac{x}{\sigma}\right) H_n\left(\frac{x}{\sigma}\right) 
e^{-\left(\frac{x}{\sigma}\right)^2}dx = 
\sigma \rtpi\, 2^m m!\, \delta_{mn}
\end{equation}

Using this in equation \ref{eqn_deltaHexp2}, we have

\begin{equation*}
H_n(0)  =  \sum_{m=0}^\infty h_m \sigma \rtpi\, 2^m m!\, \delta_{mn} \nonumber
\end{equation*}
which gives
\begin{equation}
h_n = \frac{H_n(0)}{\sigma \rtpi \, 2^n \, n!}
\label{eqn_deltaHexp_hn}
\end{equation}
One thing that can be observed by this definition is that for all odd values of $n$, the
coefficient $h_n$ is zero because $H_n(0)=0$ for all odd-order Hermite polynomials. 
This makes sense when we consider that the Dirac delta function is symmetric in $x$, and
therefore only even Hermite functions, i.e. those that are also symmetric in $x$, will
contribute to the expansion series. 

Note that we can use the Hermite polynomial recurrsion relation, given as
\begin{equation}
\label{eqn_HermiteRecur}
H_{n+1}(x) = 2xH_n(x)-2nH_{n-1}(x)
\end{equation}
to quickly find the coefficients. We can see that the first several $h_n$  are
\begin{eqnarray}
h_0 & =&  1 \nonumber \\
h_1 &=& 0 \nonumber \\
h_2 &=& 2*0*H_1(0)-2*1*H_0(0) = -2 \nonumber \\
h_3 &=& 0 \nonumber \\
h_4 &=& 2*0*H_3(0) - 2*3*-2= 12 \nonumber \\
&...& \nonumber
\end{eqnarray}

%%%%%%%%%%%%%%%%%%%%%%%%%%%%%%%%%%%%%%%%%%%%%%%%%%%%%%%%%%%%%%%%%%%%%%%%%%%%%%%%%%%%%%%%%%
\section{Application}
Here we assess some test cases to show how the DAF method functions in practice. 

\subsection{One-Dimensional}

\subsection{Multi-Dimensional}
In a multi-dimensional application, we use the DAF to approximate a function such as
\begin{equation}
f(x,y,z) = \int\int\int f(x,y,z)\delta(x)\delta(y)\delta(z) {\rm d}x {\rm d}y {\rm d}z
\end{equation}
or in fector notation for simplicity: 
\begin{equation}
f({\bf r}) = \int_{\bf V} f({\bf r})\delta({\bf r}){\rm d}{\bf V}
\end{equation}
where $\delta({\bf r}) = \delta(r_1)\delta(r_2)\delta(r_3)...$ and ${\bf V}$ is the
multi-dimensional volume to which the DAF is being applied. 

In three spacial dimensions, these coordinates are typically x, y, and z, but the method is
not limited to strictly spacial coordinates, nor is it limited to the number of
dimensions. 

NOTE: Toeplitz Matrices?

\include{appendix}

\bibliographystyle{apsrev}
\bibliography{biblist}

\end{document}
